\chapter{Simulating Rigid Solids and Surfaces}

Simulations involving a rigid solid or surface can be performed in constant volume ensembles
(i.e., NVT and GCMC). 
In addition to the files described in Chapter \label{ch:input_files}, 
the following files are required to run a simulation with a rigid solid or surface:

\begin{itemize}
\item a molecular connectivity file with force field parameters for each atom in the solid (*.mcf)
\item a fragment library file listing the coordinates of each atom in the solid (*.dat)
\item a configuration file with the initial coordinates of the all atoms in the system (*.xyz) 
\end{itemize} 

The mcf and fragment library file can be created using the scripts discussed in Sections \label{sec:mcf} and \label{sec:library_setup}. Each of these scripts requires a starting PDB configuration file.
 
\section{PDB file}\label{sec:solid_pdb}
Since the solid structure will be rigid, no bond information is required in the 
PDB file (i.e., no lines beginning with keyword CONECT). A PDB configuration file can be created in the following ways: 

\begin{itemize}
\item Manually, with atomic coordinates from the literature. 
For example, the atomic coordinates of a Silicalite zeolite are published in Ref. \cite{Meier:1981}.

\item From a Crystallographic Information File (CIF), which can be downloaded from
the Database of Zeolite Structures (http://www.iza-structure.org/databases/).
From the home page, click on the menu "Framework Type" and select your zeolite. 
The website will display structural information about the zeolite and will have 
a link to download a CIF. The CIF contains information about the 
zeolite structure such as cell parameters, space groups, T and O atom coordinates. 
A CIF can be converted into a PDB file using either Mercury or VESTA, both of which
are available as a free download. 
For example, using VESTA:

Step 1:
From the File menu, click Open. Then download the CIF (e.g. MFI.cif)

Step 2:
From the Objects menu, click Boundary. Fill the ranges for the fractional 
coordinates according the size you want. To output a single unit cell, enter -0 to 1
in the x, y and z ranges. 
The unit cell of MFI contains 288 atoms. 
To output a 2x2x2 crystal, enter -1 to 1 in the x, y and z ranges. 
This supercell will contain 8x288 = 2304 atoms.

Step 3:
From the File menu, click Export Data. Enter a filename ending with .pdb (e.g. MFI8uc.pdb for a structure of MFI with 8 unit cells)
\end{itemize}

\section{Molecular connectivity file}\label{sec:solid_mcf}
You are going to need the PDB file and information about the solid force field you are going to use in the simulation.
In literature, you can observe two approaches in order to get the solid force field. 
In one of them the values for the interaction parameters for each atom of the solid were 
tuning for a given mixing rule in order to best fit a set of data. 
Some authors also uses to set the interaction parameters for the solid equal to zero and 
then tuning the cross interactions parameters in order to best fit a set of data.

In the case of MFI there are a few force field available in literature

The way to convert the PDB file in a MCF file for a solid or surface 
follows the same procedure described for components in a fluid phase. See \ref{utility:mcfgen}

\section{Getting XYZ file}\label{sec:solid_xyz}

Once again there is many ways to convert the PDB file in a XYZ file. 
You can always do your own script or you can use free software like VMD.

Reference of VMD:
Humphrey, W., Dalke, A. and Schulten, K.
VMD - Visual Molecular Dynamics
J. Molec. Graphics 1996, 14.1, 33-38

\texttt{
> vmd "name".pdb \\
> set all [atomselect top all] \\
> \$all writexyz "name".xyz \\
}
\\ \\

Example:

\texttt{
> vmd MFI8uc.pdb \\
> set all [atomselect top all] \\
> \$all writexyz MFI8uc.xyz \\
}
\\ \\
Since the XYZ file created by VMD informs the number of atoms and Cassandra needs a 
file with the number of molecules of each species you will need to modify the xyz file generated by VMD. 
For detailed information about to convert that file in the one that  will be used for starting a simulation 
from an empty solid using the option Read\_Old see /Chapter4/Start/\_Type  T

Done. You have created a XYZ file in a proper format to be used in Cassandra.

\section{Preparing the input file}

At this point, you should have already prepared the MCF and XYZ files following the procedures described above. \\ 

Now you are going to prepare the input file for running a short simulation in order to generate the 
Fragment files for both components (the solid and the fluid). However, 
you will come across with the question: What are the proper values should 
I put in \texttt{Chemical\_Potential\_Info}?. This keyword "Chemical\_Potential\_Info" sets 
the chemical potential of the insertable species in the order in which they 
appear in the \texttt{Molecule\_Files} section. \\ 

The chemical potentials will be set arbitrarily to zero for species that cannot 
be swapped or exchanged with a reservoir. Note that the de Broglie wavelength of the 
species is automatically calculated and used in the acceptance rules (see \ref{sec:MuVT}). \\

Units of chemical potential are kJ/mol. Then, since you are not going to insert/remove any 
solid, you just need to inform the chemical potential for the other species 
(in this example: methane). Then, you will need to run an independent GCMC 
simulation for the fluid phase in order to get the appropriated value for chemical 
potential (See example in /Examples/GCMC/Methane/README). \\

Other important question that you can come across is about the move 
probabilities \texttt{Move\_Probability\_Info}. 
Remember that the solid does not have freedom of move. 
Then all the probabilities involving the solid must be set equal to zero. \\

Now you are ready to obtain the Fragment\_Files. \\

\section{Getting the Fragment Files}\label{sec:fragment file}
Since the solid or surface is considered to be rigid there is just one possible conformation 
for the atoms. You can use the script provided in \ref{utility:libgen} to
automatically generate the files for the \texttt{Fragment\_Files}.
The script will take information from the PDB file. \\


Done. You are ready for running your simulation.
_______________________________________________________________________
For example, an adsorption isotherm can be computed with a GCMC simulation in which fluid molecules
are inserted into a crystalline solid.

The simulation box dimensions is related to the geometry and size of the solid, which is kept constant. 

In  order to illustrate how to run GCMC simulation involving a rigid solid we are presenting an example 
for adsorption of methane in Silicalite (MFI) at 300K. 
A very detailed information about this example is available in Examples/GCMC/Methane\_Silicalite/README

