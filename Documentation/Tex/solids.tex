\chapter{Simulating Rigid Solids and Surfaces}

Simulations involving a rigid solid or surface can be performed in constant volume ensembles
(i.e., NVT and GCMC). 
For example, an adsorption isotherm can be computed with a GCMC simulation in which fluid molecules
are inserted into a crystalline solid.
In addition to the files described in Chapter \ref{ch:input_files}, 
the following files are required to run a simulation with a rigid solid or surface:

\begin{itemize}
\item a molecular connectivity file with force field parameters for each atom in the solid (*.mcf)
\item a fragment library file listing the coordinates of each atom in the solid (*.dat)
\item a configuration file with the initial coordinates of the all atoms in the system (*.xyz) 
\end{itemize} 

The MCF and fragment library file can be created using the scripts discussed in Sections \ref{sec:mcfgen} and \ref{sec:libgen}. 
Each of these scripts requires a starting PDB configuration file.
The input file also requires specific parameters as given in the following section. Sample input scripts for GCMC simulations
of various fluids in silicalite are included in the Examples/GCMC directory of the Cassandra distribution.
 
\section{Input file}
The input file should be completed as described in Section \ref{sec:Input_File}, but with the following parameters:

\begin{itemize}
	\item Under the keyword \texttt{\# Prob\_Translation}, the translation width for the solid is 0.
	\item Under the keyword \texttt{\# Prob\_Rotation}, the rotation width for the solid is 0.
	\item Under the keyword \texttt{\# Prob\_Regrowth}, the regrowth probability for the solid is 0.
	\item The volume dimensions under the keyword \texttt{\# Box\_Info} must match the crystal dimensions.
	\item Under the keyword \texttt{\# Start\_Type}, the \texttt{read\_config} or \texttt{add\_to\_config} options must be used.
\end{itemize}

GCMC simulations require the following additional parameters:

\begin{itemize}
	\item Under the keyword \texttt{\# Prob\_Insertion} insertion method for the solid is \texttt{none}.
	\item Under the keyword \texttt{\# Chemical\_Potential\_Info}, no chemical potential is required for the solid.
\end{itemize}

\section{PDB file}\label{sec:solid_pdb}
A PDB configuration file for a zeolite can be created in the following ways, among others: 

\begin{itemize}
	\item Manually, with atomic coordinates from the literature. 
	For example, the atomic coordinates of silicalite are published in Ref. \cite{Meier:1981}.

	\item From a Crystallographic Information File (CIF), which can be downloaded from
	the Database of Zeolite Structures (http://www.iza-structure.org/databases/).
	From the home page, click on the menu "Framework Type" and select your zeolite. 
	The website will display structural information about the zeolite and will have 
	a link to download a CIF. The CIF contains information about the 
	zeolite structure such as cell parameters, space groups, T and O atom coordinates. 
	A CIF can be converted into a PDB file using either Mercury or VESTA, both of which
	are available to download for free.
	For example, using VESTA:

	\begin{enumerate}
		\item From the File menu, click Open. Then download the CIF (e.g. MFI.cif)

		\item From the Objects menu, click Boundary. Enter the desired number of replicas
		along each axis. To output a single unit cell, enter -0 to 1 in the x, y and z ranges. 
		To output a 2x2x2 crystal, enter -1 to 1 in the x, y and z ranges. 

		\item From the File menu, click Export Data. Enter a filename ending with .pdb (e.g. MFI.pdb)
	\end{enumerate}

\end{itemize}

Executing the script \texttt{mcfgen.py} for solids involves using the 
\texttt{-s} or \texttt{--solid}. This will skip the CONECT information
that might be present in PDB files of solid materials, and thus
skip creating connectivity information such as bonds, angles, 
dihedrals and fragments.

\section{Molecular connectivity file}\label{sec:solid_mcf}
Since the solid structure will be rigid, no bond distances, angle parameters or dihedral parameters are needed in the MCF.
The PDB file for the rigid solid does not list CONECT information, so the \texttt{mcfgen.py} script will not include bond, 
angle, or dihedral sections in the force field template (*.ff) or MCF. The number of fragments will be zero. 
Only nonbonded parameters are needed.

\section{Fragment library files}\label{sec:fragment file}
The \texttt{library\_setup.py} script will not create a fragment library since the MCF lists zero fragments.

\section{Configuration xyz file}\label{sec:solid_xyz}
A simulation is initiated from an xyz file using the \texttt{read\_config} or \texttt{add\_to\_config} options. 
Section \ref{sec:Start_Type} details the \texttt{read\_config} and \texttt{add\_to\_config} options.
