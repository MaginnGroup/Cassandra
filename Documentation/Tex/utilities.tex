\chapter{Utilities}

\section{Lowenstein Script} \label{Sec:Lowenstein}
This is a script provided for easy augmentation of a given silicate structure subject to the 
Lowenstein Rule for zeolites.  Given a single unit cell of arbitrary shape, this script randomly inserts Aluminum
atoms until the specified Lowenstein Ratio (\# Si / \# Al) is achieved.  Acceptable target ratios are on the range
of [1.0, inf).

\subsection{Usage} \label{sec:Usage}
To use the Lowenstein\_Script.f90: compile the script, open the executable, and run the program.
Sample code would be:

\texttt{> gfortran Lowenstein\_Script.f90 -o Lowenstein\_Script; ./Lowenstein\_Script} \\

Following this, the user is presented with four sequential prompts:

\texttt{> Enter the name of the initial data file:}  \\
\texttt{> Enter the desired final Lowenstein Ratio ( [1.0, inf)): } \\
\texttt{> Use random initial seed? (Y/N): }\\
\texttt{> Enter the name of the output file: } \\


To reproduce a structure with an identical Lowenstein ratio as example\_ratio\_2\_1.xyz (which has a Lowenstein ratio 
of 2), one might respond to the prompts with:

\texttt{> Enter the name of the initial data file:  example.pdb  }\\
\texttt{> Enter the desired final Lowenstein Ratio ( [1.0, inf)):  2.0  }\\
\texttt{> Use random initial seed? (Y/N): Y  } \\
\texttt{> Enter the name of the output file: output\_file.xyz  } \\

The file output\_file.xyz will contain your new structure with a Lowenstein Ratio of 2.0 and will 
be placed in the current directory.


\subsection{Input} \label{sec:Input}
The script reads the first line of the .pdb according to the following format code:

\begin{lstlisting}[firstnumber=184, caption=Lowenstein\_Script.f90]
    350 FORMAT(8X,2(F8.4,1X),F8.4,3(F7.3))
\end{lstlisting}

This data is stored in variables corresponding to the A, B, C, $\alpha$, $\beta$, and $\gamma$   
parameters of a unit cell and is required for the script to function.

The next line of code in the file should contain information pertaining to atom number 1 of the unit cell, and 
all other atoms should immediately follow on distinct lines, uninterrupted until the 'END' statement of the file is reached.

Information for each atom that should be present on their respective line includes:
\begin{itemize}
\item x-coordinate
\item y-coordinate
\item z-coordinate
\item element
\end{itemize}

The format code that reads in this information is given by:

\begin{lstlisting}[firstnumber=231, caption=Lowenstein\_Script.f90]
    450 FORMAT(31X,3(F7.3,1X),21X,A2)
\end{lstlisting}

For a direct example of acceptable input format, see:  /Scripts/Lowenstein/example.pdb

\section{Input File GUI} \label{Sec:GUI}
This script provides a graphical user interface with which the user may create input files for use in simulations.  The GUI is written
 in Python and uses the wxWidgets for Python module, which may be found at:   

http://wxpython.org/download.php 



